
% Default to the notebook output style

    


% Inherit from the specified cell style.




    
\documentclass[11pt]{article}

    
    
    \usepackage[T1]{fontenc}
    % Nicer default font (+ math font) than Computer Modern for most use cases
    \usepackage{mathpazo}

    % Basic figure setup, for now with no caption control since it's done
    % automatically by Pandoc (which extracts ![](path) syntax from Markdown).
    \usepackage{graphicx}
    % We will generate all images so they have a width \maxwidth. This means
    % that they will get their normal width if they fit onto the page, but
    % are scaled down if they would overflow the margins.
    \makeatletter
    \def\maxwidth{\ifdim\Gin@nat@width>\linewidth\linewidth
    \else\Gin@nat@width\fi}
    \makeatother
    \let\Oldincludegraphics\includegraphics
    % Set max figure width to be 80% of text width, for now hardcoded.
    \renewcommand{\includegraphics}[1]{\Oldincludegraphics[width=.8\maxwidth]{#1}}
    % Ensure that by default, figures have no caption (until we provide a
    % proper Figure object with a Caption API and a way to capture that
    % in the conversion process - todo).
    \usepackage{caption}
    \DeclareCaptionLabelFormat{nolabel}{}
    \captionsetup{labelformat=nolabel}

    \usepackage{adjustbox} % Used to constrain images to a maximum size 
    \usepackage{xcolor} % Allow colors to be defined
    \usepackage{enumerate} % Needed for markdown enumerations to work
    \usepackage{geometry} % Used to adjust the document margins
    \usepackage{amsmath} % Equations
    \usepackage{amssymb} % Equations
    \usepackage{textcomp} % defines textquotesingle
    % Hack from http://tex.stackexchange.com/a/47451/13684:
    \AtBeginDocument{%
        \def\PYZsq{\textquotesingle}% Upright quotes in Pygmentized code
    }
    \usepackage{upquote} % Upright quotes for verbatim code
    \usepackage{eurosym} % defines \euro
    \usepackage[mathletters]{ucs} % Extended unicode (utf-8) support
    \usepackage[utf8x]{inputenc} % Allow utf-8 characters in the tex document
    \usepackage{fancyvrb} % verbatim replacement that allows latex
    \usepackage{grffile} % extends the file name processing of package graphics 
                         % to support a larger range 
    % The hyperref package gives us a pdf with properly built
    % internal navigation ('pdf bookmarks' for the table of contents,
    % internal cross-reference links, web links for URLs, etc.)
    \usepackage{hyperref}
    \usepackage{longtable} % longtable support required by pandoc >1.10
    \usepackage{booktabs}  % table support for pandoc > 1.12.2
    \usepackage[inline]{enumitem} % IRkernel/repr support (it uses the enumerate* environment)
    \usepackage[normalem]{ulem} % ulem is needed to support strikethroughs (\sout)
                                % normalem makes italics be italics, not underlines
    

    
    
    % Colors for the hyperref package
    \definecolor{urlcolor}{rgb}{0,.145,.698}
    \definecolor{linkcolor}{rgb}{.71,0.21,0.01}
    \definecolor{citecolor}{rgb}{.12,.54,.11}

    % ANSI colors
    \definecolor{ansi-black}{HTML}{3E424D}
    \definecolor{ansi-black-intense}{HTML}{282C36}
    \definecolor{ansi-red}{HTML}{E75C58}
    \definecolor{ansi-red-intense}{HTML}{B22B31}
    \definecolor{ansi-green}{HTML}{00A250}
    \definecolor{ansi-green-intense}{HTML}{007427}
    \definecolor{ansi-yellow}{HTML}{DDB62B}
    \definecolor{ansi-yellow-intense}{HTML}{B27D12}
    \definecolor{ansi-blue}{HTML}{208FFB}
    \definecolor{ansi-blue-intense}{HTML}{0065CA}
    \definecolor{ansi-magenta}{HTML}{D160C4}
    \definecolor{ansi-magenta-intense}{HTML}{A03196}
    \definecolor{ansi-cyan}{HTML}{60C6C8}
    \definecolor{ansi-cyan-intense}{HTML}{258F8F}
    \definecolor{ansi-white}{HTML}{C5C1B4}
    \definecolor{ansi-white-intense}{HTML}{A1A6B2}

    % commands and environments needed by pandoc snippets
    % extracted from the output of `pandoc -s`
    \providecommand{\tightlist}{%
      \setlength{\itemsep}{0pt}\setlength{\parskip}{0pt}}
    \DefineVerbatimEnvironment{Highlighting}{Verbatim}{commandchars=\\\{\}}
    % Add ',fontsize=\small' for more characters per line
    \newenvironment{Shaded}{}{}
    \newcommand{\KeywordTok}[1]{\textcolor[rgb]{0.00,0.44,0.13}{\textbf{{#1}}}}
    \newcommand{\DataTypeTok}[1]{\textcolor[rgb]{0.56,0.13,0.00}{{#1}}}
    \newcommand{\DecValTok}[1]{\textcolor[rgb]{0.25,0.63,0.44}{{#1}}}
    \newcommand{\BaseNTok}[1]{\textcolor[rgb]{0.25,0.63,0.44}{{#1}}}
    \newcommand{\FloatTok}[1]{\textcolor[rgb]{0.25,0.63,0.44}{{#1}}}
    \newcommand{\CharTok}[1]{\textcolor[rgb]{0.25,0.44,0.63}{{#1}}}
    \newcommand{\StringTok}[1]{\textcolor[rgb]{0.25,0.44,0.63}{{#1}}}
    \newcommand{\CommentTok}[1]{\textcolor[rgb]{0.38,0.63,0.69}{\textit{{#1}}}}
    \newcommand{\OtherTok}[1]{\textcolor[rgb]{0.00,0.44,0.13}{{#1}}}
    \newcommand{\AlertTok}[1]{\textcolor[rgb]{1.00,0.00,0.00}{\textbf{{#1}}}}
    \newcommand{\FunctionTok}[1]{\textcolor[rgb]{0.02,0.16,0.49}{{#1}}}
    \newcommand{\RegionMarkerTok}[1]{{#1}}
    \newcommand{\ErrorTok}[1]{\textcolor[rgb]{1.00,0.00,0.00}{\textbf{{#1}}}}
    \newcommand{\NormalTok}[1]{{#1}}
    
    % Additional commands for more recent versions of Pandoc
    \newcommand{\ConstantTok}[1]{\textcolor[rgb]{0.53,0.00,0.00}{{#1}}}
    \newcommand{\SpecialCharTok}[1]{\textcolor[rgb]{0.25,0.44,0.63}{{#1}}}
    \newcommand{\VerbatimStringTok}[1]{\textcolor[rgb]{0.25,0.44,0.63}{{#1}}}
    \newcommand{\SpecialStringTok}[1]{\textcolor[rgb]{0.73,0.40,0.53}{{#1}}}
    \newcommand{\ImportTok}[1]{{#1}}
    \newcommand{\DocumentationTok}[1]{\textcolor[rgb]{0.73,0.13,0.13}{\textit{{#1}}}}
    \newcommand{\AnnotationTok}[1]{\textcolor[rgb]{0.38,0.63,0.69}{\textbf{\textit{{#1}}}}}
    \newcommand{\CommentVarTok}[1]{\textcolor[rgb]{0.38,0.63,0.69}{\textbf{\textit{{#1}}}}}
    \newcommand{\VariableTok}[1]{\textcolor[rgb]{0.10,0.09,0.49}{{#1}}}
    \newcommand{\ControlFlowTok}[1]{\textcolor[rgb]{0.00,0.44,0.13}{\textbf{{#1}}}}
    \newcommand{\OperatorTok}[1]{\textcolor[rgb]{0.40,0.40,0.40}{{#1}}}
    \newcommand{\BuiltInTok}[1]{{#1}}
    \newcommand{\ExtensionTok}[1]{{#1}}
    \newcommand{\PreprocessorTok}[1]{\textcolor[rgb]{0.74,0.48,0.00}{{#1}}}
    \newcommand{\AttributeTok}[1]{\textcolor[rgb]{0.49,0.56,0.16}{{#1}}}
    \newcommand{\InformationTok}[1]{\textcolor[rgb]{0.38,0.63,0.69}{\textbf{\textit{{#1}}}}}
    \newcommand{\WarningTok}[1]{\textcolor[rgb]{0.38,0.63,0.69}{\textbf{\textit{{#1}}}}}
    
    
    % Define a nice break command that doesn't care if a line doesn't already
    % exist.
    \def\br{\hspace*{\fill} \\* }
    % Math Jax compatability definitions
    \def\gt{>}
    \def\lt{<}
    % Document parameters
    \title{Clustering}
    
    
    

    % Pygments definitions
    
\makeatletter
\def\PY@reset{\let\PY@it=\relax \let\PY@bf=\relax%
    \let\PY@ul=\relax \let\PY@tc=\relax%
    \let\PY@bc=\relax \let\PY@ff=\relax}
\def\PY@tok#1{\csname PY@tok@#1\endcsname}
\def\PY@toks#1+{\ifx\relax#1\empty\else%
    \PY@tok{#1}\expandafter\PY@toks\fi}
\def\PY@do#1{\PY@bc{\PY@tc{\PY@ul{%
    \PY@it{\PY@bf{\PY@ff{#1}}}}}}}
\def\PY#1#2{\PY@reset\PY@toks#1+\relax+\PY@do{#2}}

\expandafter\def\csname PY@tok@w\endcsname{\def\PY@tc##1{\textcolor[rgb]{0.73,0.73,0.73}{##1}}}
\expandafter\def\csname PY@tok@c\endcsname{\let\PY@it=\textit\def\PY@tc##1{\textcolor[rgb]{0.25,0.50,0.50}{##1}}}
\expandafter\def\csname PY@tok@cp\endcsname{\def\PY@tc##1{\textcolor[rgb]{0.74,0.48,0.00}{##1}}}
\expandafter\def\csname PY@tok@k\endcsname{\let\PY@bf=\textbf\def\PY@tc##1{\textcolor[rgb]{0.00,0.50,0.00}{##1}}}
\expandafter\def\csname PY@tok@kp\endcsname{\def\PY@tc##1{\textcolor[rgb]{0.00,0.50,0.00}{##1}}}
\expandafter\def\csname PY@tok@kt\endcsname{\def\PY@tc##1{\textcolor[rgb]{0.69,0.00,0.25}{##1}}}
\expandafter\def\csname PY@tok@o\endcsname{\def\PY@tc##1{\textcolor[rgb]{0.40,0.40,0.40}{##1}}}
\expandafter\def\csname PY@tok@ow\endcsname{\let\PY@bf=\textbf\def\PY@tc##1{\textcolor[rgb]{0.67,0.13,1.00}{##1}}}
\expandafter\def\csname PY@tok@nb\endcsname{\def\PY@tc##1{\textcolor[rgb]{0.00,0.50,0.00}{##1}}}
\expandafter\def\csname PY@tok@nf\endcsname{\def\PY@tc##1{\textcolor[rgb]{0.00,0.00,1.00}{##1}}}
\expandafter\def\csname PY@tok@nc\endcsname{\let\PY@bf=\textbf\def\PY@tc##1{\textcolor[rgb]{0.00,0.00,1.00}{##1}}}
\expandafter\def\csname PY@tok@nn\endcsname{\let\PY@bf=\textbf\def\PY@tc##1{\textcolor[rgb]{0.00,0.00,1.00}{##1}}}
\expandafter\def\csname PY@tok@ne\endcsname{\let\PY@bf=\textbf\def\PY@tc##1{\textcolor[rgb]{0.82,0.25,0.23}{##1}}}
\expandafter\def\csname PY@tok@nv\endcsname{\def\PY@tc##1{\textcolor[rgb]{0.10,0.09,0.49}{##1}}}
\expandafter\def\csname PY@tok@no\endcsname{\def\PY@tc##1{\textcolor[rgb]{0.53,0.00,0.00}{##1}}}
\expandafter\def\csname PY@tok@nl\endcsname{\def\PY@tc##1{\textcolor[rgb]{0.63,0.63,0.00}{##1}}}
\expandafter\def\csname PY@tok@ni\endcsname{\let\PY@bf=\textbf\def\PY@tc##1{\textcolor[rgb]{0.60,0.60,0.60}{##1}}}
\expandafter\def\csname PY@tok@na\endcsname{\def\PY@tc##1{\textcolor[rgb]{0.49,0.56,0.16}{##1}}}
\expandafter\def\csname PY@tok@nt\endcsname{\let\PY@bf=\textbf\def\PY@tc##1{\textcolor[rgb]{0.00,0.50,0.00}{##1}}}
\expandafter\def\csname PY@tok@nd\endcsname{\def\PY@tc##1{\textcolor[rgb]{0.67,0.13,1.00}{##1}}}
\expandafter\def\csname PY@tok@s\endcsname{\def\PY@tc##1{\textcolor[rgb]{0.73,0.13,0.13}{##1}}}
\expandafter\def\csname PY@tok@sd\endcsname{\let\PY@it=\textit\def\PY@tc##1{\textcolor[rgb]{0.73,0.13,0.13}{##1}}}
\expandafter\def\csname PY@tok@si\endcsname{\let\PY@bf=\textbf\def\PY@tc##1{\textcolor[rgb]{0.73,0.40,0.53}{##1}}}
\expandafter\def\csname PY@tok@se\endcsname{\let\PY@bf=\textbf\def\PY@tc##1{\textcolor[rgb]{0.73,0.40,0.13}{##1}}}
\expandafter\def\csname PY@tok@sr\endcsname{\def\PY@tc##1{\textcolor[rgb]{0.73,0.40,0.53}{##1}}}
\expandafter\def\csname PY@tok@ss\endcsname{\def\PY@tc##1{\textcolor[rgb]{0.10,0.09,0.49}{##1}}}
\expandafter\def\csname PY@tok@sx\endcsname{\def\PY@tc##1{\textcolor[rgb]{0.00,0.50,0.00}{##1}}}
\expandafter\def\csname PY@tok@m\endcsname{\def\PY@tc##1{\textcolor[rgb]{0.40,0.40,0.40}{##1}}}
\expandafter\def\csname PY@tok@gh\endcsname{\let\PY@bf=\textbf\def\PY@tc##1{\textcolor[rgb]{0.00,0.00,0.50}{##1}}}
\expandafter\def\csname PY@tok@gu\endcsname{\let\PY@bf=\textbf\def\PY@tc##1{\textcolor[rgb]{0.50,0.00,0.50}{##1}}}
\expandafter\def\csname PY@tok@gd\endcsname{\def\PY@tc##1{\textcolor[rgb]{0.63,0.00,0.00}{##1}}}
\expandafter\def\csname PY@tok@gi\endcsname{\def\PY@tc##1{\textcolor[rgb]{0.00,0.63,0.00}{##1}}}
\expandafter\def\csname PY@tok@gr\endcsname{\def\PY@tc##1{\textcolor[rgb]{1.00,0.00,0.00}{##1}}}
\expandafter\def\csname PY@tok@ge\endcsname{\let\PY@it=\textit}
\expandafter\def\csname PY@tok@gs\endcsname{\let\PY@bf=\textbf}
\expandafter\def\csname PY@tok@gp\endcsname{\let\PY@bf=\textbf\def\PY@tc##1{\textcolor[rgb]{0.00,0.00,0.50}{##1}}}
\expandafter\def\csname PY@tok@go\endcsname{\def\PY@tc##1{\textcolor[rgb]{0.53,0.53,0.53}{##1}}}
\expandafter\def\csname PY@tok@gt\endcsname{\def\PY@tc##1{\textcolor[rgb]{0.00,0.27,0.87}{##1}}}
\expandafter\def\csname PY@tok@err\endcsname{\def\PY@bc##1{\setlength{\fboxsep}{0pt}\fcolorbox[rgb]{1.00,0.00,0.00}{1,1,1}{\strut ##1}}}
\expandafter\def\csname PY@tok@kc\endcsname{\let\PY@bf=\textbf\def\PY@tc##1{\textcolor[rgb]{0.00,0.50,0.00}{##1}}}
\expandafter\def\csname PY@tok@kd\endcsname{\let\PY@bf=\textbf\def\PY@tc##1{\textcolor[rgb]{0.00,0.50,0.00}{##1}}}
\expandafter\def\csname PY@tok@kn\endcsname{\let\PY@bf=\textbf\def\PY@tc##1{\textcolor[rgb]{0.00,0.50,0.00}{##1}}}
\expandafter\def\csname PY@tok@kr\endcsname{\let\PY@bf=\textbf\def\PY@tc##1{\textcolor[rgb]{0.00,0.50,0.00}{##1}}}
\expandafter\def\csname PY@tok@bp\endcsname{\def\PY@tc##1{\textcolor[rgb]{0.00,0.50,0.00}{##1}}}
\expandafter\def\csname PY@tok@fm\endcsname{\def\PY@tc##1{\textcolor[rgb]{0.00,0.00,1.00}{##1}}}
\expandafter\def\csname PY@tok@vc\endcsname{\def\PY@tc##1{\textcolor[rgb]{0.10,0.09,0.49}{##1}}}
\expandafter\def\csname PY@tok@vg\endcsname{\def\PY@tc##1{\textcolor[rgb]{0.10,0.09,0.49}{##1}}}
\expandafter\def\csname PY@tok@vi\endcsname{\def\PY@tc##1{\textcolor[rgb]{0.10,0.09,0.49}{##1}}}
\expandafter\def\csname PY@tok@vm\endcsname{\def\PY@tc##1{\textcolor[rgb]{0.10,0.09,0.49}{##1}}}
\expandafter\def\csname PY@tok@sa\endcsname{\def\PY@tc##1{\textcolor[rgb]{0.73,0.13,0.13}{##1}}}
\expandafter\def\csname PY@tok@sb\endcsname{\def\PY@tc##1{\textcolor[rgb]{0.73,0.13,0.13}{##1}}}
\expandafter\def\csname PY@tok@sc\endcsname{\def\PY@tc##1{\textcolor[rgb]{0.73,0.13,0.13}{##1}}}
\expandafter\def\csname PY@tok@dl\endcsname{\def\PY@tc##1{\textcolor[rgb]{0.73,0.13,0.13}{##1}}}
\expandafter\def\csname PY@tok@s2\endcsname{\def\PY@tc##1{\textcolor[rgb]{0.73,0.13,0.13}{##1}}}
\expandafter\def\csname PY@tok@sh\endcsname{\def\PY@tc##1{\textcolor[rgb]{0.73,0.13,0.13}{##1}}}
\expandafter\def\csname PY@tok@s1\endcsname{\def\PY@tc##1{\textcolor[rgb]{0.73,0.13,0.13}{##1}}}
\expandafter\def\csname PY@tok@mb\endcsname{\def\PY@tc##1{\textcolor[rgb]{0.40,0.40,0.40}{##1}}}
\expandafter\def\csname PY@tok@mf\endcsname{\def\PY@tc##1{\textcolor[rgb]{0.40,0.40,0.40}{##1}}}
\expandafter\def\csname PY@tok@mh\endcsname{\def\PY@tc##1{\textcolor[rgb]{0.40,0.40,0.40}{##1}}}
\expandafter\def\csname PY@tok@mi\endcsname{\def\PY@tc##1{\textcolor[rgb]{0.40,0.40,0.40}{##1}}}
\expandafter\def\csname PY@tok@il\endcsname{\def\PY@tc##1{\textcolor[rgb]{0.40,0.40,0.40}{##1}}}
\expandafter\def\csname PY@tok@mo\endcsname{\def\PY@tc##1{\textcolor[rgb]{0.40,0.40,0.40}{##1}}}
\expandafter\def\csname PY@tok@ch\endcsname{\let\PY@it=\textit\def\PY@tc##1{\textcolor[rgb]{0.25,0.50,0.50}{##1}}}
\expandafter\def\csname PY@tok@cm\endcsname{\let\PY@it=\textit\def\PY@tc##1{\textcolor[rgb]{0.25,0.50,0.50}{##1}}}
\expandafter\def\csname PY@tok@cpf\endcsname{\let\PY@it=\textit\def\PY@tc##1{\textcolor[rgb]{0.25,0.50,0.50}{##1}}}
\expandafter\def\csname PY@tok@c1\endcsname{\let\PY@it=\textit\def\PY@tc##1{\textcolor[rgb]{0.25,0.50,0.50}{##1}}}
\expandafter\def\csname PY@tok@cs\endcsname{\let\PY@it=\textit\def\PY@tc##1{\textcolor[rgb]{0.25,0.50,0.50}{##1}}}

\def\PYZbs{\char`\\}
\def\PYZus{\char`\_}
\def\PYZob{\char`\{}
\def\PYZcb{\char`\}}
\def\PYZca{\char`\^}
\def\PYZam{\char`\&}
\def\PYZlt{\char`\<}
\def\PYZgt{\char`\>}
\def\PYZsh{\char`\#}
\def\PYZpc{\char`\%}
\def\PYZdl{\char`\$}
\def\PYZhy{\char`\-}
\def\PYZsq{\char`\'}
\def\PYZdq{\char`\"}
\def\PYZti{\char`\~}
% for compatibility with earlier versions
\def\PYZat{@}
\def\PYZlb{[}
\def\PYZrb{]}
\makeatother


    % Exact colors from NB
    \definecolor{incolor}{rgb}{0.0, 0.0, 0.5}
    \definecolor{outcolor}{rgb}{0.545, 0.0, 0.0}



    
    % Prevent overflowing lines due to hard-to-break entities
    \sloppy 
    % Setup hyperref package
    \hypersetup{
      breaklinks=true,  % so long urls are correctly broken across lines
      colorlinks=true,
      urlcolor=urlcolor,
      linkcolor=linkcolor,
      citecolor=citecolor,
      }
    % Slightly bigger margins than the latex defaults
    
    \geometry{verbose,tmargin=1in,bmargin=1in,lmargin=1in,rmargin=1in}
    
    

    \begin{document}
    
    
    \maketitle
    
    

    
    \hypertarget{unsupervised-learning}{%
\section{Unsupervised learning}\label{unsupervised-learning}}

    Unsupervised learning is a method used to enable machines to classify
objects without providing the machines any prior information about the
objects. The main idea behind unsupervised learning is to expose the
machines to large volumes of varied data and allow it to learn and infer
from the data. However, the machines must first be programmed to learn
from data.

    \hypertarget{clustering}{%
\subsection{1. Clustering}\label{clustering}}

    Clustering is the task of dividing the population or data points into a
number of groups such that data points in the same groups are more
similar to other data points in the same group than those in other
groups. The aim is to segregate groups with similar traits and assign
them into clusters.

    \hypertarget{hierarchical-clustering}{%
\subsubsection{1.1. Hierarchical
clustering}\label{hierarchical-clustering}}

    Hierarchical clustering, also known as hierarchical cluster analysis, is
an algorithm that groups similar objects into groups called clusters.
The endpoint is a set of clusters, where each cluster is distinct from
each other cluster, and the objects within each cluster are broadly
similar to each other.\\
This clustering technique is divided into two types:\\
1.Agglomerative\\
2.Divisive

    \hypertarget{agglomerative-hierarchical-clustering}{%
\paragraph{1.1.1. Agglomerative Hierarchical
clustering}\label{agglomerative-hierarchical-clustering}}

    Agglomerative Hierarchical clustering Technique: In this technique,
initially each data point is considered as an individual cluster. At
each iteration, the similar clusters merge with other clusters until one
cluster or K clusters are formed. The basic algorithm of Agglomerative
is straight forward.\\
•Compute the proximity matrix\\
•Let each data point be a cluster\\
•Repeat: Merge the two closest clusters and update the proximity
matrix\\
•Until only a single cluster remains

    \hypertarget{divisive-hierarchical-clustering}{%
\paragraph{1.1.2. Divisive Hierarchical
clustering}\label{divisive-hierarchical-clustering}}

    Divisive Hierarchical clustering Technique: Divisive Hierarchical
clustering is exactly the opposite of the Agglomerative Hierarchical
clustering. In Divisive Hierarchical clustering, we consider all the
data points as a single cluster and in each iteration, we separate the
data points from the cluster which are not similar. Each data point
which is separated is considered as an individual cluster. In the end,
we'll be left with n clusters.

    Calculating the similarity between two clusters is important to merge or
divide the clusters. There are different ways to find distance between
the clusters. Following are some of the options to measure distance
between two clusters:\\
1.Measure the distance between the closest points of two clusters.\\
2.Measure the distance between the farthest points of two clusters.\\
3.Measure the distance between the centroids of two clusters.\\
4.Measure the distance between all possible combination of points
between the two clusters and take the mean.

    The Hierarchical clustering Technique can be visualized using a
Dendrogram. A Dendrogram is a tree-like diagram that records the
sequences of merges or splits.

    Once one large cluster is formed by the combination of small clusters,
dendrograms of the cluster are used to actually split the cluster into
multiple clusters of related data points. Let's see how it's actually
done.

Suppose we have a collection of data points represented by a numpy array
as follows:

    \begin{Verbatim}[commandchars=\\\{\}]
{\color{incolor}In [{\color{incolor}50}]:} \PY{k+kn}{import} \PY{n+nn}{numpy} \PY{k}{as} \PY{n+nn}{np}
         \PY{n}{X} \PY{o}{=} \PY{n}{np}\PY{o}{.}\PY{n}{array}\PY{p}{(}\PY{p}{[}\PY{p}{[}\PY{l+m+mi}{3}\PY{p}{,}\PY{l+m+mi}{5}\PY{p}{]}\PY{p}{,}
             \PY{p}{[}\PY{l+m+mi}{10}\PY{p}{,}\PY{l+m+mi}{18}\PY{p}{]}\PY{p}{,}
             \PY{p}{[}\PY{l+m+mi}{14}\PY{p}{,}\PY{l+m+mi}{12}\PY{p}{]}\PY{p}{,}
             \PY{p}{[}\PY{l+m+mi}{24}\PY{p}{,}\PY{l+m+mi}{8}\PY{p}{]}\PY{p}{,}
             \PY{p}{[}\PY{l+m+mi}{30}\PY{p}{,}\PY{l+m+mi}{30}\PY{p}{]}\PY{p}{,}
             \PY{p}{[}\PY{l+m+mi}{85}\PY{p}{,}\PY{l+m+mi}{77}\PY{p}{]}\PY{p}{,}
             \PY{p}{[}\PY{l+m+mi}{74}\PY{p}{,}\PY{l+m+mi}{83}\PY{p}{]}\PY{p}{,}
             \PY{p}{[}\PY{l+m+mi}{62}\PY{p}{,}\PY{l+m+mi}{78}\PY{p}{]}\PY{p}{,}
             \PY{p}{[}\PY{l+m+mi}{70}\PY{p}{,}\PY{l+m+mi}{58}\PY{p}{]}\PY{p}{,}
             \PY{p}{[}\PY{l+m+mi}{80}\PY{p}{,}\PY{l+m+mi}{90}\PY{p}{]}\PY{p}{,}\PY{p}{]}\PY{p}{)}
\end{Verbatim}

    \begin{Verbatim}[commandchars=\\\{\}]
{\color{incolor}In [{\color{incolor}51}]:} \PY{k+kn}{import} \PY{n+nn}{matplotlib}\PY{n+nn}{.}\PY{n+nn}{pyplot} \PY{k}{as} \PY{n+nn}{plt}
         
         \PY{n}{labels} \PY{o}{=} \PY{n+nb}{range}\PY{p}{(}\PY{l+m+mi}{1}\PY{p}{,} \PY{l+m+mi}{11}\PY{p}{)}
         \PY{n}{plt}\PY{o}{.}\PY{n}{figure}\PY{p}{(}\PY{n}{figsize}\PY{o}{=}\PY{p}{(}\PY{l+m+mi}{10}\PY{p}{,} \PY{l+m+mi}{7}\PY{p}{)}\PY{p}{)}
         \PY{n}{plt}\PY{o}{.}\PY{n}{subplots\PYZus{}adjust}\PY{p}{(}\PY{n}{bottom}\PY{o}{=}\PY{l+m+mf}{0.1}\PY{p}{)}
         \PY{n}{plt}\PY{o}{.}\PY{n}{scatter}\PY{p}{(}\PY{n}{X}\PY{p}{[}\PY{p}{:}\PY{p}{,}\PY{l+m+mi}{0}\PY{p}{]}\PY{p}{,}\PY{n}{X}\PY{p}{[}\PY{p}{:}\PY{p}{,}\PY{l+m+mi}{1}\PY{p}{]}\PY{p}{,} \PY{n}{label}\PY{o}{=}\PY{l+s+s1}{\PYZsq{}}\PY{l+s+s1}{True Position}\PY{l+s+s1}{\PYZsq{}}\PY{p}{)}
         
         \PY{k}{for} \PY{n}{label}\PY{p}{,} \PY{n}{x}\PY{p}{,} \PY{n}{y} \PY{o+ow}{in} \PY{n+nb}{zip}\PY{p}{(}\PY{n}{labels}\PY{p}{,} \PY{n}{X}\PY{p}{[}\PY{p}{:}\PY{p}{,} \PY{l+m+mi}{0}\PY{p}{]}\PY{p}{,} \PY{n}{X}\PY{p}{[}\PY{p}{:}\PY{p}{,} \PY{l+m+mi}{1}\PY{p}{]}\PY{p}{)}\PY{p}{:}
             \PY{n}{plt}\PY{o}{.}\PY{n}{annotate}\PY{p}{(}
                 \PY{n}{label}\PY{p}{,}
                 \PY{n}{xy}\PY{o}{=}\PY{p}{(}\PY{n}{x}\PY{p}{,} \PY{n}{y}\PY{p}{)}\PY{p}{,} \PY{n}{xytext}\PY{o}{=}\PY{p}{(}\PY{o}{\PYZhy{}}\PY{l+m+mi}{3}\PY{p}{,} \PY{l+m+mi}{3}\PY{p}{)}\PY{p}{,}
                 \PY{n}{textcoords}\PY{o}{=}\PY{l+s+s1}{\PYZsq{}}\PY{l+s+s1}{offset points}\PY{l+s+s1}{\PYZsq{}}\PY{p}{,} \PY{n}{ha}\PY{o}{=}\PY{l+s+s1}{\PYZsq{}}\PY{l+s+s1}{right}\PY{l+s+s1}{\PYZsq{}}\PY{p}{,} \PY{n}{va}\PY{o}{=}\PY{l+s+s1}{\PYZsq{}}\PY{l+s+s1}{bottom}\PY{l+s+s1}{\PYZsq{}}\PY{p}{)}
         \PY{n}{plt}\PY{o}{.}\PY{n}{show}\PY{p}{(}\PY{p}{)}
\end{Verbatim}

    \begin{center}
    \adjustimage{max size={0.9\linewidth}{0.9\paperheight}}{Clustering_files/Clustering_14_0.png}
    \end{center}
    { \hspace*{\fill} \\}
    
    It can be seen from the naked eye that the data points form two
clusters: first at the bottom left consisting of points 1-5 while second
at the top right consisting of points 6-10.

    However, in the real world, we may have thousands of data points in many
more than 2 dimensions. In that case it would not be possible to spot
clusters with the naked eye. This is why clustering algorithms have been
developed.

Let's draw the dendrograms for our data points. We will use the scipy
library for that purpose.

    \begin{Verbatim}[commandchars=\\\{\}]
{\color{incolor}In [{\color{incolor}52}]:} \PY{k+kn}{from} \PY{n+nn}{scipy}\PY{n+nn}{.}\PY{n+nn}{cluster}\PY{n+nn}{.}\PY{n+nn}{hierarchy} \PY{k}{import} \PY{n}{dendrogram}\PY{p}{,} \PY{n}{linkage}
         \PY{k+kn}{from} \PY{n+nn}{matplotlib} \PY{k}{import} \PY{n}{pyplot} \PY{k}{as} \PY{n}{plt}
         
         \PY{n}{linked} \PY{o}{=} \PY{n}{linkage}\PY{p}{(}\PY{n}{X}\PY{p}{,} \PY{l+s+s1}{\PYZsq{}}\PY{l+s+s1}{single}\PY{l+s+s1}{\PYZsq{}}\PY{p}{)}
         
         \PY{n}{labelList} \PY{o}{=} \PY{n+nb}{range}\PY{p}{(}\PY{l+m+mi}{1}\PY{p}{,} \PY{l+m+mi}{11}\PY{p}{)}
         
         \PY{n}{plt}\PY{o}{.}\PY{n}{figure}\PY{p}{(}\PY{n}{figsize}\PY{o}{=}\PY{p}{(}\PY{l+m+mi}{10}\PY{p}{,} \PY{l+m+mi}{7}\PY{p}{)}\PY{p}{)}
         \PY{n}{plt}\PY{o}{.}\PY{n}{title}\PY{p}{(}\PY{l+s+s1}{\PYZsq{}}\PY{l+s+s1}{Hierarchical Clustering Dendrogram}\PY{l+s+s1}{\PYZsq{}}\PY{p}{)}
         \PY{n}{plt}\PY{o}{.}\PY{n}{xlabel}\PY{p}{(}\PY{l+s+s1}{\PYZsq{}}\PY{l+s+s1}{Index}\PY{l+s+s1}{\PYZsq{}}\PY{p}{)}
         \PY{n}{plt}\PY{o}{.}\PY{n}{ylabel}\PY{p}{(}\PY{l+s+s1}{\PYZsq{}}\PY{l+s+s1}{distance}\PY{l+s+s1}{\PYZsq{}}\PY{p}{)}
         \PY{n}{dendrogram}\PY{p}{(}\PY{n}{linked}\PY{p}{,}
                     \PY{n}{orientation}\PY{o}{=}\PY{l+s+s1}{\PYZsq{}}\PY{l+s+s1}{top}\PY{l+s+s1}{\PYZsq{}}\PY{p}{,}
                     \PY{n}{labels}\PY{o}{=}\PY{n}{labelList}\PY{p}{,}
                     \PY{n}{distance\PYZus{}sort}\PY{o}{=}\PY{l+s+s1}{\PYZsq{}}\PY{l+s+s1}{descending}\PY{l+s+s1}{\PYZsq{}}\PY{p}{,}
                     \PY{n}{show\PYZus{}leaf\PYZus{}counts}\PY{o}{=}\PY{k+kc}{True}\PY{p}{)}
         \PY{n}{plt}\PY{o}{.}\PY{n}{show}\PY{p}{(}\PY{p}{)}
\end{Verbatim}

    \begin{center}
    \adjustimage{max size={0.9\linewidth}{0.9\paperheight}}{Clustering_files/Clustering_17_0.png}
    \end{center}
    { \hspace*{\fill} \\}
    
    \hypertarget{dendrogram-truncation}{%
\subparagraph{Dendrogram Truncation}\label{dendrogram-truncation}}

    We can truncate dendrogram according to the application of data, we can
find last p clusters or can trucate it at any specific distance. Let's
draw dendrogram of last 5 clusters of our data points.

    \begin{Verbatim}[commandchars=\\\{\}]
{\color{incolor}In [{\color{incolor}53}]:} \PY{n}{plt}\PY{o}{.}\PY{n}{figure}\PY{p}{(}\PY{n}{figsize}\PY{o}{=}\PY{p}{(}\PY{l+m+mi}{10}\PY{p}{,} \PY{l+m+mi}{7}\PY{p}{)}\PY{p}{)}
         \PY{n}{plt}\PY{o}{.}\PY{n}{title}\PY{p}{(}\PY{l+s+s1}{\PYZsq{}}\PY{l+s+s1}{Hierarchical Clustering Dendrogram (Truncated)}\PY{l+s+s1}{\PYZsq{}}\PY{p}{)}
         \PY{n}{plt}\PY{o}{.}\PY{n}{xlabel}\PY{p}{(}\PY{l+s+s1}{\PYZsq{}}\PY{l+s+s1}{Index}\PY{l+s+s1}{\PYZsq{}}\PY{p}{)}
         \PY{n}{plt}\PY{o}{.}\PY{n}{ylabel}\PY{p}{(}\PY{l+s+s1}{\PYZsq{}}\PY{l+s+s1}{distance}\PY{l+s+s1}{\PYZsq{}}\PY{p}{)}
         \PY{n}{dendrogram}\PY{p}{(}\PY{n}{linked}\PY{p}{,}
                     \PY{n}{orientation}\PY{o}{=}\PY{l+s+s1}{\PYZsq{}}\PY{l+s+s1}{top}\PY{l+s+s1}{\PYZsq{}}\PY{p}{,}
                     \PY{n}{labels}\PY{o}{=}\PY{n}{labelList}\PY{p}{,}
                     \PY{n}{distance\PYZus{}sort}\PY{o}{=}\PY{l+s+s1}{\PYZsq{}}\PY{l+s+s1}{descending}\PY{l+s+s1}{\PYZsq{}}\PY{p}{,}
                     \PY{n}{truncate\PYZus{}mode}\PY{o}{=}\PY{l+s+s1}{\PYZsq{}}\PY{l+s+s1}{lastp}\PY{l+s+s1}{\PYZsq{}}\PY{p}{,}  \PY{c+c1}{\PYZsh{} show only the last p merged clusters}
                     \PY{n}{p}\PY{o}{=}\PY{l+m+mi}{5}\PY{p}{,}  \PY{c+c1}{\PYZsh{} show only the last p merged clusters}
                     \PY{n}{show\PYZus{}leaf\PYZus{}counts}\PY{o}{=}\PY{k+kc}{True}\PY{p}{)}
         \PY{n}{plt}\PY{o}{.}\PY{n}{show}\PY{p}{(}\PY{p}{)}
\end{Verbatim}

    \begin{center}
    \adjustimage{max size={0.9\linewidth}{0.9\paperheight}}{Clustering_files/Clustering_20_0.png}
    \end{center}
    { \hspace*{\fill} \\}
    
    Above plot shows us last 5 clusters of our data points.

    Now let's implement hierarchical clustering using Python's Scikit-Learn
library.

    \begin{Verbatim}[commandchars=\\\{\}]
{\color{incolor}In [{\color{incolor}54}]:} \PY{k+kn}{from} \PY{n+nn}{sklearn}\PY{n+nn}{.}\PY{n+nn}{cluster} \PY{k}{import} \PY{n}{AgglomerativeClustering}
         
         \PY{n}{cluster} \PY{o}{=} \PY{n}{AgglomerativeClustering}\PY{p}{(}\PY{n}{n\PYZus{}clusters}\PY{o}{=}\PY{l+m+mi}{2}\PY{p}{,} \PY{n}{affinity}\PY{o}{=}\PY{l+s+s1}{\PYZsq{}}\PY{l+s+s1}{euclidean}\PY{l+s+s1}{\PYZsq{}}\PY{p}{,} \PY{n}{linkage}\PY{o}{=}\PY{l+s+s1}{\PYZsq{}}\PY{l+s+s1}{ward}\PY{l+s+s1}{\PYZsq{}}\PY{p}{)}
         \PY{n}{cluster}\PY{o}{.}\PY{n}{fit\PYZus{}predict}\PY{p}{(}\PY{n}{X}\PY{p}{)}
         \PY{n+nb}{print}\PY{p}{(}\PY{n}{cluster}\PY{o}{.}\PY{n}{labels\PYZus{}}\PY{p}{)}
         \PY{n}{plt}\PY{o}{.}\PY{n}{scatter}\PY{p}{(}\PY{n}{X}\PY{p}{[}\PY{p}{:}\PY{p}{,}\PY{l+m+mi}{0}\PY{p}{]}\PY{p}{,}\PY{n}{X}\PY{p}{[}\PY{p}{:}\PY{p}{,}\PY{l+m+mi}{1}\PY{p}{]}\PY{p}{,} \PY{n}{c}\PY{o}{=}\PY{n}{cluster}\PY{o}{.}\PY{n}{labels\PYZus{}}\PY{p}{,} \PY{n}{cmap}\PY{o}{=}\PY{l+s+s1}{\PYZsq{}}\PY{l+s+s1}{rainbow}\PY{l+s+s1}{\PYZsq{}}\PY{p}{)}
\end{Verbatim}

    \begin{Verbatim}[commandchars=\\\{\}]
[0 0 0 0 0 1 1 1 1 1]

    \end{Verbatim}

            \begin{Verbatim}[commandchars=\\\{\}]
{\color{outcolor}Out[{\color{outcolor}54}]:} <matplotlib.collections.PathCollection at 0x2828e4912e8>
\end{Verbatim}
        
    \begin{center}
    \adjustimage{max size={0.9\linewidth}{0.9\paperheight}}{Clustering_files/Clustering_23_2.png}
    \end{center}
    { \hspace*{\fill} \\}
    
    We can see points in two clusters where the first five points clustered
together and the last five points clustered together.

    Advantages of Hierarchical clustering Technique:\\
1.No apriori information about the number of clusters required.\\
2.Easy to implement and gives best result in some cases.

    Limitations of Hierarchical clustering Technique:\\
1.There is no mathematical objective for Hierarchical clustering.\\
2.All the approaches to calculate the similarity between clusters has
its own disadvantages.\\
3.High space and time complexity for Hierarchical clustering. Hence this
clustering algorithm cannot be used when we have huge data.

    \hypertarget{k-means}{%
\subsubsection{1.2. k-Means}\label{k-means}}

    The k-means algorithm searches for a pre-determined number of clusters
within an unlabeled multidimensional dataset. It accomplishes this using
a simple conception of what the optimal clustering looks like:\\
•The ``cluster center'' is the arithmetic mean of all the points
belonging to the cluster.\\
•Each point is closer to its own cluster center than to other cluster
centers.

Those two assumptions are the basis of the k-means model. Let's take a
look at a simple dataset and see the k-means result.

First, let's generate a two-dimensional dataset containing four distinct
blobs. To emphasize that this is an unsupervised algorithm, we will
leave the labels out of the visualization

    \begin{Verbatim}[commandchars=\\\{\}]
{\color{incolor}In [{\color{incolor}55}]:} \PY{o}{\PYZpc{}}\PY{k}{matplotlib} inline
         \PY{k+kn}{import} \PY{n+nn}{matplotlib}\PY{n+nn}{.}\PY{n+nn}{pyplot} \PY{k}{as} \PY{n+nn}{plt}
         \PY{k+kn}{import} \PY{n+nn}{seaborn} \PY{k}{as} \PY{n+nn}{sns}\PY{p}{;} \PY{n}{sns}\PY{o}{.}\PY{n}{set}\PY{p}{(}\PY{p}{)}  \PY{c+c1}{\PYZsh{} for plot styling}
         \PY{k+kn}{import} \PY{n+nn}{numpy} \PY{k}{as} \PY{n+nn}{np}
\end{Verbatim}

    \begin{Verbatim}[commandchars=\\\{\}]
{\color{incolor}In [{\color{incolor}56}]:} \PY{k+kn}{from} \PY{n+nn}{sklearn}\PY{n+nn}{.}\PY{n+nn}{datasets}\PY{n+nn}{.}\PY{n+nn}{samples\PYZus{}generator} \PY{k}{import} \PY{n}{make\PYZus{}blobs}
         \PY{n}{X}\PY{p}{,} \PY{n}{y\PYZus{}true} \PY{o}{=} \PY{n}{make\PYZus{}blobs}\PY{p}{(}\PY{n}{n\PYZus{}samples}\PY{o}{=}\PY{l+m+mi}{300}\PY{p}{,} \PY{n}{centers}\PY{o}{=}\PY{l+m+mi}{4}\PY{p}{,}
                                \PY{n}{cluster\PYZus{}std}\PY{o}{=}\PY{l+m+mf}{0.60}\PY{p}{,} \PY{n}{random\PYZus{}state}\PY{o}{=}\PY{l+m+mi}{0}\PY{p}{)}
         \PY{n}{plt}\PY{o}{.}\PY{n}{scatter}\PY{p}{(}\PY{n}{X}\PY{p}{[}\PY{p}{:}\PY{p}{,} \PY{l+m+mi}{0}\PY{p}{]}\PY{p}{,} \PY{n}{X}\PY{p}{[}\PY{p}{:}\PY{p}{,} \PY{l+m+mi}{1}\PY{p}{]}\PY{p}{,} \PY{n}{s}\PY{o}{=}\PY{l+m+mi}{50}\PY{p}{)}\PY{p}{;}
\end{Verbatim}

    \begin{center}
    \adjustimage{max size={0.9\linewidth}{0.9\paperheight}}{Clustering_files/Clustering_30_0.png}
    \end{center}
    { \hspace*{\fill} \\}
    
    By eye, it is relatively easy to pick out the four clusters. The k-means
algorithm does this automatically, and in Scikit-Learn uses the typical
estimator API:

    \begin{Verbatim}[commandchars=\\\{\}]
{\color{incolor}In [{\color{incolor}57}]:} \PY{k+kn}{from} \PY{n+nn}{sklearn}\PY{n+nn}{.}\PY{n+nn}{cluster} \PY{k}{import} \PY{n}{KMeans}
         \PY{n}{kmeans} \PY{o}{=} \PY{n}{KMeans}\PY{p}{(}\PY{n}{n\PYZus{}clusters}\PY{o}{=}\PY{l+m+mi}{4}\PY{p}{)}
         \PY{n}{kmeans}\PY{o}{.}\PY{n}{fit}\PY{p}{(}\PY{n}{X}\PY{p}{)}
         \PY{n}{y\PYZus{}kmeans} \PY{o}{=} \PY{n}{kmeans}\PY{o}{.}\PY{n}{predict}\PY{p}{(}\PY{n}{X}\PY{p}{)}
\end{Verbatim}

    \begin{Verbatim}[commandchars=\\\{\}]
{\color{incolor}In [{\color{incolor}58}]:} \PY{n}{plt}\PY{o}{.}\PY{n}{scatter}\PY{p}{(}\PY{n}{X}\PY{p}{[}\PY{p}{:}\PY{p}{,} \PY{l+m+mi}{0}\PY{p}{]}\PY{p}{,} \PY{n}{X}\PY{p}{[}\PY{p}{:}\PY{p}{,} \PY{l+m+mi}{1}\PY{p}{]}\PY{p}{,} \PY{n}{c}\PY{o}{=}\PY{n}{y\PYZus{}kmeans}\PY{p}{,} \PY{n}{s}\PY{o}{=}\PY{l+m+mi}{50}\PY{p}{,} \PY{n}{cmap}\PY{o}{=}\PY{l+s+s1}{\PYZsq{}}\PY{l+s+s1}{viridis}\PY{l+s+s1}{\PYZsq{}}\PY{p}{)}
         
         \PY{n}{centers} \PY{o}{=} \PY{n}{kmeans}\PY{o}{.}\PY{n}{cluster\PYZus{}centers\PYZus{}}
         \PY{n}{plt}\PY{o}{.}\PY{n}{scatter}\PY{p}{(}\PY{n}{centers}\PY{p}{[}\PY{p}{:}\PY{p}{,} \PY{l+m+mi}{0}\PY{p}{]}\PY{p}{,} \PY{n}{centers}\PY{p}{[}\PY{p}{:}\PY{p}{,} \PY{l+m+mi}{1}\PY{p}{]}\PY{p}{,} \PY{n}{c}\PY{o}{=}\PY{l+s+s1}{\PYZsq{}}\PY{l+s+s1}{black}\PY{l+s+s1}{\PYZsq{}}\PY{p}{,} \PY{n}{s}\PY{o}{=}\PY{l+m+mi}{100}\PY{p}{,} \PY{n}{alpha}\PY{o}{=}\PY{l+m+mf}{0.7}\PY{p}{)}\PY{p}{;}
\end{Verbatim}

    \begin{center}
    \adjustimage{max size={0.9\linewidth}{0.9\paperheight}}{Clustering_files/Clustering_33_0.png}
    \end{center}
    { \hspace*{\fill} \\}
    
    How this algorithm finds these clusters so quickly! After all, the
number of possible combinations of cluster assignments is exponential in
the number of data points---an exhaustive search would be very, very
costly, but such an exhaustive search is not necessary: instead, the
typical approach to k-means involves an intuitive iterative approach
known as expectation--maximization.

    \hypertarget{k-means-algorithm-expectationmaximization}{%
\paragraph{k-Means Algorithm:
Expectation--Maximization}\label{k-means-algorithm-expectationmaximization}}

    Expectation--maximization (E--M) is a powerful algorithm that comes up
in a variety of contexts within data science. k-means is a particularly
simple and easy-to-understand application of the algorithm, and we will
walk through it briefly here. In short, the expectation--maximization
approach here consists of the following procedure:\\
1.Guess some cluster centers\\
2.Repeat until converged

\begin{verbatim}
1.E-Step: assign points to the nearest cluster center
2.M-Step: set the cluster centers to the mean 
\end{verbatim}

Here the ``E-step'' or ``Expectation step'' is so-named because it
involves updating our expectation of which cluster each point belongs
to. The ``M-step'' or ``Maximization step'' is so-named because it
involves maximizing some fitness function that defines the location of
the cluster centers---in this case, that maximization is accomplished by
taking a simple mean of the data in each cluster.

The literature about this algorithm is vast, but can be summarized as
follows: under typical circumstances, each repetition of the E-step and
M-step will always result in a better estimate of the cluster
characteristics.

    The k-Means algorithm is simple enough that we can write it in a few
lines of code. The following is a very basic implementation:

    \begin{Verbatim}[commandchars=\\\{\}]
{\color{incolor}In [{\color{incolor}59}]:} \PY{k+kn}{from} \PY{n+nn}{sklearn}\PY{n+nn}{.}\PY{n+nn}{metrics} \PY{k}{import} \PY{n}{pairwise\PYZus{}distances\PYZus{}argmin}
         
         \PY{k}{def} \PY{n+nf}{find\PYZus{}clusters}\PY{p}{(}\PY{n}{X}\PY{p}{,} \PY{n}{n\PYZus{}clusters}\PY{p}{,} \PY{n}{rseed}\PY{o}{=}\PY{l+m+mi}{2}\PY{p}{)}\PY{p}{:}
             \PY{c+c1}{\PYZsh{} 1. Randomly choose clusters}
             \PY{n}{rng} \PY{o}{=} \PY{n}{np}\PY{o}{.}\PY{n}{random}\PY{o}{.}\PY{n}{RandomState}\PY{p}{(}\PY{n}{rseed}\PY{p}{)}
             \PY{n}{i} \PY{o}{=} \PY{n}{rng}\PY{o}{.}\PY{n}{permutation}\PY{p}{(}\PY{n}{X}\PY{o}{.}\PY{n}{shape}\PY{p}{[}\PY{l+m+mi}{0}\PY{p}{]}\PY{p}{)}\PY{p}{[}\PY{p}{:}\PY{n}{n\PYZus{}clusters}\PY{p}{]}
             \PY{n}{centers} \PY{o}{=} \PY{n}{X}\PY{p}{[}\PY{n}{i}\PY{p}{]}
             
             \PY{k}{while} \PY{k+kc}{True}\PY{p}{:}
                 \PY{c+c1}{\PYZsh{} 2a. Assign labels based on closest center}
                 \PY{n}{labels} \PY{o}{=} \PY{n}{pairwise\PYZus{}distances\PYZus{}argmin}\PY{p}{(}\PY{n}{X}\PY{p}{,} \PY{n}{centers}\PY{p}{)}
                 
                 \PY{c+c1}{\PYZsh{} 2b. Find new centers from means of points}
                 \PY{n}{new\PYZus{}centers} \PY{o}{=} \PY{n}{np}\PY{o}{.}\PY{n}{array}\PY{p}{(}\PY{p}{[}\PY{n}{X}\PY{p}{[}\PY{n}{labels} \PY{o}{==} \PY{n}{i}\PY{p}{]}\PY{o}{.}\PY{n}{mean}\PY{p}{(}\PY{l+m+mi}{0}\PY{p}{)}
                                         \PY{k}{for} \PY{n}{i} \PY{o+ow}{in} \PY{n+nb}{range}\PY{p}{(}\PY{n}{n\PYZus{}clusters}\PY{p}{)}\PY{p}{]}\PY{p}{)}
                 
                 \PY{c+c1}{\PYZsh{} 2c. Check for convergence}
                 \PY{k}{if} \PY{n}{np}\PY{o}{.}\PY{n}{all}\PY{p}{(}\PY{n}{centers} \PY{o}{==} \PY{n}{new\PYZus{}centers}\PY{p}{)}\PY{p}{:}
                     \PY{k}{break}
                 \PY{n}{centers} \PY{o}{=} \PY{n}{new\PYZus{}centers}
             
             \PY{k}{return} \PY{n}{centers}\PY{p}{,} \PY{n}{labels}
         
         \PY{n}{centers}\PY{p}{,} \PY{n}{labels} \PY{o}{=} \PY{n}{find\PYZus{}clusters}\PY{p}{(}\PY{n}{X}\PY{p}{,} \PY{l+m+mi}{4}\PY{p}{)}
         \PY{n}{plt}\PY{o}{.}\PY{n}{scatter}\PY{p}{(}\PY{n}{X}\PY{p}{[}\PY{p}{:}\PY{p}{,} \PY{l+m+mi}{0}\PY{p}{]}\PY{p}{,} \PY{n}{X}\PY{p}{[}\PY{p}{:}\PY{p}{,} \PY{l+m+mi}{1}\PY{p}{]}\PY{p}{,} \PY{n}{c}\PY{o}{=}\PY{n}{labels}\PY{p}{,}
                     \PY{n}{s}\PY{o}{=}\PY{l+m+mi}{50}\PY{p}{,} \PY{n}{cmap}\PY{o}{=}\PY{l+s+s1}{\PYZsq{}}\PY{l+s+s1}{viridis}\PY{l+s+s1}{\PYZsq{}}\PY{p}{)}\PY{p}{;}
\end{Verbatim}

    \begin{center}
    \adjustimage{max size={0.9\linewidth}{0.9\paperheight}}{Clustering_files/Clustering_38_0.png}
    \end{center}
    { \hspace*{\fill} \\}
    
    There are a few issues to be aware of when using the
expectation--maximization algorithm.

First, although the E--M procedure is guaranteed to improve the result
in each step, there is no assurance that it will lead to the global best
solution. For example, if we use a different random seed in our simple
procedure, the particular starting guesses lead to poor results:

    \begin{Verbatim}[commandchars=\\\{\}]
{\color{incolor}In [{\color{incolor}60}]:} \PY{n}{centers}\PY{p}{,} \PY{n}{labels} \PY{o}{=} \PY{n}{find\PYZus{}clusters}\PY{p}{(}\PY{n}{X}\PY{p}{,} \PY{l+m+mi}{4}\PY{p}{,} \PY{n}{rseed}\PY{o}{=}\PY{l+m+mi}{3}\PY{p}{)}
         \PY{n}{plt}\PY{o}{.}\PY{n}{scatter}\PY{p}{(}\PY{n}{X}\PY{p}{[}\PY{p}{:}\PY{p}{,} \PY{l+m+mi}{0}\PY{p}{]}\PY{p}{,} \PY{n}{X}\PY{p}{[}\PY{p}{:}\PY{p}{,} \PY{l+m+mi}{1}\PY{p}{]}\PY{p}{,} \PY{n}{c}\PY{o}{=}\PY{n}{labels}\PY{p}{,}
                     \PY{n}{s}\PY{o}{=}\PY{l+m+mi}{50}\PY{p}{,} \PY{n}{cmap}\PY{o}{=}\PY{l+s+s1}{\PYZsq{}}\PY{l+s+s1}{viridis}\PY{l+s+s1}{\PYZsq{}}\PY{p}{)}\PY{p}{;}
\end{Verbatim}

    \begin{center}
    \adjustimage{max size={0.9\linewidth}{0.9\paperheight}}{Clustering_files/Clustering_40_0.png}
    \end{center}
    { \hspace*{\fill} \\}
    
    Here the E--M approach has converged, but has not converged to a
globally optimal configuration. For this reason, it is common for the
algorithm to be run for multiple starting guesses, as indeed
Scikit-Learn does by default (set by the n\_init parameter, which
defaults to 10).

    Another common challenge with k-means is that we must tell it how many
clusters we expect: it cannot learn the number of clusters from the
data. For example, if we ask the algorithm to identify six clusters, it
will happily proceed and find the best six clusters:

    \begin{Verbatim}[commandchars=\\\{\}]
{\color{incolor}In [{\color{incolor}61}]:} \PY{n}{labels} \PY{o}{=} \PY{n}{KMeans}\PY{p}{(}\PY{l+m+mi}{6}\PY{p}{,} \PY{n}{random\PYZus{}state}\PY{o}{=}\PY{l+m+mi}{0}\PY{p}{)}\PY{o}{.}\PY{n}{fit\PYZus{}predict}\PY{p}{(}\PY{n}{X}\PY{p}{)}
         \PY{n}{plt}\PY{o}{.}\PY{n}{scatter}\PY{p}{(}\PY{n}{X}\PY{p}{[}\PY{p}{:}\PY{p}{,} \PY{l+m+mi}{0}\PY{p}{]}\PY{p}{,} \PY{n}{X}\PY{p}{[}\PY{p}{:}\PY{p}{,} \PY{l+m+mi}{1}\PY{p}{]}\PY{p}{,} \PY{n}{c}\PY{o}{=}\PY{n}{labels}\PY{p}{,}
                     \PY{n}{s}\PY{o}{=}\PY{l+m+mi}{50}\PY{p}{,} \PY{n}{cmap}\PY{o}{=}\PY{l+s+s1}{\PYZsq{}}\PY{l+s+s1}{viridis}\PY{l+s+s1}{\PYZsq{}}\PY{p}{)}\PY{p}{;}
\end{Verbatim}

    \begin{center}
    \adjustimage{max size={0.9\linewidth}{0.9\paperheight}}{Clustering_files/Clustering_43_0.png}
    \end{center}
    { \hspace*{\fill} \\}
    
    The fundamental model assumptions of k-means (points will be closer to
their own cluster center than to others) means that the algorithm will
often be ineffective if the clusters have complicated geometries.

    An important observation for k-means is that these cluster models must
be circular: k-means has no built-in way of accounting for oblong or
elliptical clusters. So, for example, if we take the same data and
transform it, the cluster assignments end up becoming muddled:

    \begin{Verbatim}[commandchars=\\\{\}]
{\color{incolor}In [{\color{incolor}62}]:} \PY{n}{rng} \PY{o}{=} \PY{n}{np}\PY{o}{.}\PY{n}{random}\PY{o}{.}\PY{n}{RandomState}\PY{p}{(}\PY{l+m+mi}{13}\PY{p}{)}
         \PY{n}{X\PYZus{}stretched} \PY{o}{=} \PY{n}{np}\PY{o}{.}\PY{n}{dot}\PY{p}{(}\PY{n}{X}\PY{p}{,} \PY{n}{rng}\PY{o}{.}\PY{n}{randn}\PY{p}{(}\PY{l+m+mi}{2}\PY{p}{,} \PY{l+m+mi}{2}\PY{p}{)}\PY{p}{)}
         
         \PY{k+kn}{from} \PY{n+nn}{sklearn}\PY{n+nn}{.}\PY{n+nn}{cluster} \PY{k}{import} \PY{n}{KMeans}
         \PY{k+kn}{from} \PY{n+nn}{scipy}\PY{n+nn}{.}\PY{n+nn}{spatial}\PY{n+nn}{.}\PY{n+nn}{distance} \PY{k}{import} \PY{n}{cdist}
         
         \PY{k}{def} \PY{n+nf}{plot\PYZus{}kmeans}\PY{p}{(}\PY{n}{kmeans}\PY{p}{,} \PY{n}{X}\PY{p}{,} \PY{n}{n\PYZus{}clusters}\PY{o}{=}\PY{l+m+mi}{4}\PY{p}{,} \PY{n}{rseed}\PY{o}{=}\PY{l+m+mi}{0}\PY{p}{,} \PY{n}{ax}\PY{o}{=}\PY{k+kc}{None}\PY{p}{)}\PY{p}{:}
             \PY{n}{labels} \PY{o}{=} \PY{n}{kmeans}\PY{o}{.}\PY{n}{fit\PYZus{}predict}\PY{p}{(}\PY{n}{X}\PY{p}{)}
         
             \PY{c+c1}{\PYZsh{} plot the input data}
             \PY{n}{ax} \PY{o}{=} \PY{n}{ax} \PY{o+ow}{or} \PY{n}{plt}\PY{o}{.}\PY{n}{gca}\PY{p}{(}\PY{p}{)}
             \PY{n}{ax}\PY{o}{.}\PY{n}{axis}\PY{p}{(}\PY{l+s+s1}{\PYZsq{}}\PY{l+s+s1}{equal}\PY{l+s+s1}{\PYZsq{}}\PY{p}{)}
             \PY{n}{ax}\PY{o}{.}\PY{n}{scatter}\PY{p}{(}\PY{n}{X}\PY{p}{[}\PY{p}{:}\PY{p}{,} \PY{l+m+mi}{0}\PY{p}{]}\PY{p}{,} \PY{n}{X}\PY{p}{[}\PY{p}{:}\PY{p}{,} \PY{l+m+mi}{1}\PY{p}{]}\PY{p}{,} \PY{n}{c}\PY{o}{=}\PY{n}{labels}\PY{p}{,} \PY{n}{s}\PY{o}{=}\PY{l+m+mi}{40}\PY{p}{,} \PY{n}{cmap}\PY{o}{=}\PY{l+s+s1}{\PYZsq{}}\PY{l+s+s1}{viridis}\PY{l+s+s1}{\PYZsq{}}\PY{p}{,} \PY{n}{zorder}\PY{o}{=}\PY{l+m+mi}{2}\PY{p}{)}
         
             \PY{c+c1}{\PYZsh{} plot the representation of the KMeans model}
             \PY{n}{centers} \PY{o}{=} \PY{n}{kmeans}\PY{o}{.}\PY{n}{cluster\PYZus{}centers\PYZus{}}
             \PY{n}{radii} \PY{o}{=} \PY{p}{[}\PY{n}{cdist}\PY{p}{(}\PY{n}{X}\PY{p}{[}\PY{n}{labels} \PY{o}{==} \PY{n}{i}\PY{p}{]}\PY{p}{,} \PY{p}{[}\PY{n}{center}\PY{p}{]}\PY{p}{)}\PY{o}{.}\PY{n}{max}\PY{p}{(}\PY{p}{)}
                      \PY{k}{for} \PY{n}{i}\PY{p}{,} \PY{n}{center} \PY{o+ow}{in} \PY{n+nb}{enumerate}\PY{p}{(}\PY{n}{centers}\PY{p}{)}\PY{p}{]}
             \PY{k}{for} \PY{n}{c}\PY{p}{,} \PY{n}{r} \PY{o+ow}{in} \PY{n+nb}{zip}\PY{p}{(}\PY{n}{centers}\PY{p}{,} \PY{n}{radii}\PY{p}{)}\PY{p}{:}
                 \PY{n}{ax}\PY{o}{.}\PY{n}{add\PYZus{}patch}\PY{p}{(}\PY{n}{plt}\PY{o}{.}\PY{n}{Circle}\PY{p}{(}\PY{n}{c}\PY{p}{,} \PY{n}{r}\PY{p}{,} \PY{n}{fc}\PY{o}{=}\PY{l+s+s1}{\PYZsq{}}\PY{l+s+s1}{\PYZsh{}CCCCCC}\PY{l+s+s1}{\PYZsq{}}\PY{p}{,} \PY{n}{lw}\PY{o}{=}\PY{l+m+mi}{3}\PY{p}{,} \PY{n}{alpha}\PY{o}{=}\PY{l+m+mf}{0.7}\PY{p}{,} \PY{n}{zorder}\PY{o}{=}\PY{l+m+mi}{1}\PY{p}{)}\PY{p}{)}
         
         \PY{n}{X}\PY{p}{,} \PY{n}{y\PYZus{}true} \PY{o}{=} \PY{n}{make\PYZus{}blobs}\PY{p}{(}\PY{n}{n\PYZus{}samples}\PY{o}{=}\PY{l+m+mi}{400}\PY{p}{,} \PY{n}{centers}\PY{o}{=}\PY{l+m+mi}{4}\PY{p}{,}
                                \PY{n}{cluster\PYZus{}std}\PY{o}{=}\PY{l+m+mf}{0.60}\PY{p}{,} \PY{n}{random\PYZus{}state}\PY{o}{=}\PY{l+m+mi}{0}\PY{p}{)}
         \PY{n}{X} \PY{o}{=} \PY{n}{X}\PY{p}{[}\PY{p}{:}\PY{p}{,} \PY{p}{:}\PY{p}{:}\PY{o}{\PYZhy{}}\PY{l+m+mi}{1}\PY{p}{]} 
                 
         \PY{n}{kmeans} \PY{o}{=} \PY{n}{KMeans}\PY{p}{(}\PY{n}{n\PYZus{}clusters}\PY{o}{=}\PY{l+m+mi}{4}\PY{p}{,} \PY{n}{random\PYZus{}state}\PY{o}{=}\PY{l+m+mi}{0}\PY{p}{)}
         \PY{n}{plot\PYZus{}kmeans}\PY{p}{(}\PY{n}{kmeans}\PY{p}{,} \PY{n}{X\PYZus{}stretched}\PY{p}{)}
\end{Verbatim}

    \begin{center}
    \adjustimage{max size={0.9\linewidth}{0.9\paperheight}}{Clustering_files/Clustering_46_0.png}
    \end{center}
    { \hspace*{\fill} \\}
    
    In particular, the boundaries between k-means clusters will always be
linear, which means that it will fail for more complicated boundaries.
Consider the following data, along with the cluster labels found by the
typical k-means approach:

    \begin{Verbatim}[commandchars=\\\{\}]
{\color{incolor}In [{\color{incolor}63}]:} \PY{k+kn}{from} \PY{n+nn}{sklearn}\PY{n+nn}{.}\PY{n+nn}{datasets} \PY{k}{import} \PY{n}{make\PYZus{}moons}
         \PY{n}{X}\PY{p}{,} \PY{n}{y} \PY{o}{=} \PY{n}{make\PYZus{}moons}\PY{p}{(}\PY{l+m+mi}{200}\PY{p}{,} \PY{n}{noise}\PY{o}{=}\PY{o}{.}\PY{l+m+mi}{05}\PY{p}{,} \PY{n}{random\PYZus{}state}\PY{o}{=}\PY{l+m+mi}{0}\PY{p}{)}
         \PY{n}{labels} \PY{o}{=} \PY{n}{KMeans}\PY{p}{(}\PY{l+m+mi}{2}\PY{p}{,} \PY{n}{random\PYZus{}state}\PY{o}{=}\PY{l+m+mi}{0}\PY{p}{)}\PY{o}{.}\PY{n}{fit\PYZus{}predict}\PY{p}{(}\PY{n}{X}\PY{p}{)}
         \PY{n}{plt}\PY{o}{.}\PY{n}{scatter}\PY{p}{(}\PY{n}{X}\PY{p}{[}\PY{p}{:}\PY{p}{,} \PY{l+m+mi}{0}\PY{p}{]}\PY{p}{,} \PY{n}{X}\PY{p}{[}\PY{p}{:}\PY{p}{,} \PY{l+m+mi}{1}\PY{p}{]}\PY{p}{,} \PY{n}{c}\PY{o}{=}\PY{n}{labels}\PY{p}{,}
                     \PY{n}{s}\PY{o}{=}\PY{l+m+mi}{50}\PY{p}{,} \PY{n}{cmap}\PY{o}{=}\PY{l+s+s1}{\PYZsq{}}\PY{l+s+s1}{viridis}\PY{l+s+s1}{\PYZsq{}}\PY{p}{)}\PY{p}{;}
\end{Verbatim}

    \begin{center}
    \adjustimage{max size={0.9\linewidth}{0.9\paperheight}}{Clustering_files/Clustering_48_0.png}
    \end{center}
    { \hspace*{\fill} \\}
    
    Advantages:\\
1.Fast, robust and easier to understand.\\
2.Relatively efficient: O(tknd), where n is number of objects, k is
number of number of clusters, d is number of dimension of each object,
and t is number of iterations. Normally, k, t, d \textless\textless{}
n.\\
3.Gives best result when data set are distinct or well separated from
each other.

    Disadvantages:\\
1.The learning algorithm requires apriori specification of the number of
cluster centers.\\
2.If there are two highly overlapping data then k-means will not be able
to resolve.\\
3.The learning algorithm is not invariant to non-linear transformations
i.e.~with different representation of data we get different results
(data represented in form of cartesian co-ordinates and polar
co-ordinates will give different results).\\
4.Euclidean distance measures can unequally weight underlying factors.\\
5.The learning algorithm provides the local optima of the squared error
function.\\
6.Randomly choosing of the cluster center cannot lead us to the fruitful
result. Pl. refer Fig.\\
7.Applicable only when mean is defined i.e.~fails for categorical
data.\\
8.Unable to handle noisy data and outliers.\\
9.Algorithm fails for non-linear data set.

    WE might imagine addressing these weaknesses by generalizing the k-means
model: for example, we could measure uncertainty in cluster assignment
by comparing the distances of each point to all cluster centers, rather
than focusing on just the closest. We might also imagine allowing the
cluster boundaries to be ellipses rather than circles, so as to account
for non-circular clusters. It turns out these are two essential
components of a different type of clustering model, Gaussian mixture
models.

    \hypertarget{gaussian-mixture-models}{%
\subsubsection{1.3. Gaussian Mixture
Models}\label{gaussian-mixture-models}}

    Gaussian Mixture models are used to find clusters in a dataset from
which we know (or assume to know) the number of clusters enclosed in
this dataset, but we do not know where these clusters are as well as how
they are shaped. Finding these clusters is the task of GMM and since we
don't have any information instead of the number of clusters, the GMM is
an unsupervised approach. To accomplish that, we try to fit a mixture of
gaussians to our dataset. That is, we try to find a number of gaussian
distributions which can be used to describe the shape of our dataset. A
critical point for the understanding is that these gaussian shaped
clusters must not be circular shaped as for instance in the KNN approach
but can have all shapes a multivariate Gaussian distribution can take.
That is, a circle can only change in its diameter whilst a GMM model can
(because of its covariance matrix) model all ellipsoid shapes as well.

    In the simplest case, GMMs can be used for finding clusters in the same
manner as k-means:

    \begin{Verbatim}[commandchars=\\\{\}]
{\color{incolor}In [{\color{incolor}64}]:} \PY{n}{X}\PY{p}{,} \PY{n}{y\PYZus{}true} \PY{o}{=} \PY{n}{make\PYZus{}blobs}\PY{p}{(}\PY{n}{n\PYZus{}samples}\PY{o}{=}\PY{l+m+mi}{400}\PY{p}{,} \PY{n}{centers}\PY{o}{=}\PY{l+m+mi}{4}\PY{p}{,}
                                \PY{n}{cluster\PYZus{}std}\PY{o}{=}\PY{l+m+mf}{0.60}\PY{p}{,} \PY{n}{random\PYZus{}state}\PY{o}{=}\PY{l+m+mi}{0}\PY{p}{)}
         \PY{n}{X} \PY{o}{=} \PY{n}{X}\PY{p}{[}\PY{p}{:}\PY{p}{,} \PY{p}{:}\PY{p}{:}\PY{o}{\PYZhy{}}\PY{l+m+mi}{1}\PY{p}{]} 
         \PY{k+kn}{from} \PY{n+nn}{sklearn}\PY{n+nn}{.}\PY{n+nn}{mixture} \PY{k}{import} \PY{n}{GaussianMixture} \PY{k}{as} \PY{n}{GMM}
         \PY{n}{gmm} \PY{o}{=} \PY{n}{GMM}\PY{p}{(}\PY{n}{n\PYZus{}components}\PY{o}{=}\PY{l+m+mi}{4}\PY{p}{)}\PY{o}{.}\PY{n}{fit}\PY{p}{(}\PY{n}{X}\PY{p}{)}
         \PY{n}{labels} \PY{o}{=} \PY{n}{gmm}\PY{o}{.}\PY{n}{predict}\PY{p}{(}\PY{n}{X}\PY{p}{)}
         \PY{n}{plt}\PY{o}{.}\PY{n}{scatter}\PY{p}{(}\PY{n}{X}\PY{p}{[}\PY{p}{:}\PY{p}{,} \PY{l+m+mi}{0}\PY{p}{]}\PY{p}{,} \PY{n}{X}\PY{p}{[}\PY{p}{:}\PY{p}{,} \PY{l+m+mi}{1}\PY{p}{]}\PY{p}{,} \PY{n}{c}\PY{o}{=}\PY{n}{labels}\PY{p}{,} \PY{n}{s}\PY{o}{=}\PY{l+m+mi}{40}\PY{p}{,} \PY{n}{cmap}\PY{o}{=}\PY{l+s+s1}{\PYZsq{}}\PY{l+s+s1}{viridis}\PY{l+s+s1}{\PYZsq{}}\PY{p}{)}\PY{p}{;}
\end{Verbatim}

    \begin{center}
    \adjustimage{max size={0.9\linewidth}{0.9\paperheight}}{Clustering_files/Clustering_55_0.png}
    \end{center}
    { \hspace*{\fill} \\}
    
    But because GMM contains a probabilistic model under the hood, it is
also possible to find probabilistic cluster assignments---in
Scikit-Learn this is done using the predict\_proba method. This returns
a matrix of size {[}n\_samples, n\_clusters{]} which measures the
probability that any point belongs to the given cluster:

    \begin{Verbatim}[commandchars=\\\{\}]
{\color{incolor}In [{\color{incolor}65}]:} \PY{n}{probs} \PY{o}{=} \PY{n}{gmm}\PY{o}{.}\PY{n}{predict\PYZus{}proba}\PY{p}{(}\PY{n}{X}\PY{p}{)}
         \PY{n+nb}{print}\PY{p}{(}\PY{n}{probs}\PY{p}{[}\PY{p}{:}\PY{l+m+mi}{10}\PY{p}{]}\PY{o}{.}\PY{n}{round}\PY{p}{(}\PY{l+m+mi}{3}\PY{p}{)}\PY{p}{)}
\end{Verbatim}

    \begin{Verbatim}[commandchars=\\\{\}]
[[0.    0.463 0.    0.537]
 [0.    0.    1.    0.   ]
 [0.    0.    1.    0.   ]
 [0.    0.    0.    1.   ]
 [0.    0.    1.    0.   ]
 [0.    1.    0.    0.   ]
 [1.    0.    0.    0.   ]
 [0.    0.    0.013 0.987]
 [0.    0.    1.    0.   ]
 [1.    0.    0.    0.   ]]

    \end{Verbatim}

    We can visualize this uncertainty by, for example, making the size of
each point proportional to the certainty of its prediction; looking at
the following figure, we can see that it is precisely the points at the
boundaries between clusters that reflect this uncertainty of cluster
assignment:

    \begin{Verbatim}[commandchars=\\\{\}]
{\color{incolor}In [{\color{incolor}66}]:} \PY{n}{size} \PY{o}{=} \PY{l+m+mi}{50} \PY{o}{*} \PY{n}{probs}\PY{o}{.}\PY{n}{max}\PY{p}{(}\PY{l+m+mi}{1}\PY{p}{)} \PY{o}{*}\PY{o}{*} \PY{l+m+mi}{2}  \PY{c+c1}{\PYZsh{} square emphasizes differences}
         \PY{n}{plt}\PY{o}{.}\PY{n}{scatter}\PY{p}{(}\PY{n}{X}\PY{p}{[}\PY{p}{:}\PY{p}{,} \PY{l+m+mi}{0}\PY{p}{]}\PY{p}{,} \PY{n}{X}\PY{p}{[}\PY{p}{:}\PY{p}{,} \PY{l+m+mi}{1}\PY{p}{]}\PY{p}{,} \PY{n}{c}\PY{o}{=}\PY{n}{labels}\PY{p}{,} \PY{n}{cmap}\PY{o}{=}\PY{l+s+s1}{\PYZsq{}}\PY{l+s+s1}{viridis}\PY{l+s+s1}{\PYZsq{}}\PY{p}{,} \PY{n}{s}\PY{o}{=}\PY{n}{size}\PY{p}{)}\PY{p}{;}
\end{Verbatim}

    \begin{center}
    \adjustimage{max size={0.9\linewidth}{0.9\paperheight}}{Clustering_files/Clustering_59_0.png}
    \end{center}
    { \hspace*{\fill} \\}
    
    Under the hood, a Gaussian mixture model is very similar to k-means: it
uses an expectation--maximization approach which qualitatively does the
following:\\
1.Choose starting guesses for the location and shape\\
2.Repeat until converged:

\begin{verbatim}
1.E-step: for each point, find weights encoding the probability of membership in each cluster
2.M-step: for each cluster, update its location, normalization, and shape based on all data points, making use of the weights
\end{verbatim}

The result of this is that each cluster is associated not with a
hard-edged sphere, but with a smooth Gaussian model. Just as in the
k-means expectation--maximization approach, this algorithm can sometimes
miss the globally optimal solution, and thus in practice multiple random
initializations are used.

Let's create a function that will help us visualize the locations and
shapes of the GMM clusters by drawing ellipses based on the GMM output:

    \begin{Verbatim}[commandchars=\\\{\}]
{\color{incolor}In [{\color{incolor}67}]:} \PY{k+kn}{from} \PY{n+nn}{matplotlib}\PY{n+nn}{.}\PY{n+nn}{patches} \PY{k}{import} \PY{n}{Ellipse}
         
         \PY{k}{def} \PY{n+nf}{draw\PYZus{}ellipse}\PY{p}{(}\PY{n}{position}\PY{p}{,} \PY{n}{covariance}\PY{p}{,} \PY{n}{ax}\PY{o}{=}\PY{k+kc}{None}\PY{p}{,} \PY{o}{*}\PY{o}{*}\PY{n}{kwargs}\PY{p}{)}\PY{p}{:}
             \PY{l+s+sd}{\PYZdq{}\PYZdq{}\PYZdq{}Draw an ellipse with a given position and covariance\PYZdq{}\PYZdq{}\PYZdq{}}
             \PY{n}{ax} \PY{o}{=} \PY{n}{ax} \PY{o+ow}{or} \PY{n}{plt}\PY{o}{.}\PY{n}{gca}\PY{p}{(}\PY{p}{)}
             
             \PY{c+c1}{\PYZsh{} Convert covariance to principal axes}
             \PY{k}{if} \PY{n}{covariance}\PY{o}{.}\PY{n}{shape} \PY{o}{==} \PY{p}{(}\PY{l+m+mi}{2}\PY{p}{,} \PY{l+m+mi}{2}\PY{p}{)}\PY{p}{:}
                 \PY{n}{U}\PY{p}{,} \PY{n}{s}\PY{p}{,} \PY{n}{Vt} \PY{o}{=} \PY{n}{np}\PY{o}{.}\PY{n}{linalg}\PY{o}{.}\PY{n}{svd}\PY{p}{(}\PY{n}{covariance}\PY{p}{)}
                 \PY{n}{angle} \PY{o}{=} \PY{n}{np}\PY{o}{.}\PY{n}{degrees}\PY{p}{(}\PY{n}{np}\PY{o}{.}\PY{n}{arctan2}\PY{p}{(}\PY{n}{U}\PY{p}{[}\PY{l+m+mi}{1}\PY{p}{,} \PY{l+m+mi}{0}\PY{p}{]}\PY{p}{,} \PY{n}{U}\PY{p}{[}\PY{l+m+mi}{0}\PY{p}{,} \PY{l+m+mi}{0}\PY{p}{]}\PY{p}{)}\PY{p}{)}
                 \PY{n}{width}\PY{p}{,} \PY{n}{height} \PY{o}{=} \PY{l+m+mi}{2} \PY{o}{*} \PY{n}{np}\PY{o}{.}\PY{n}{sqrt}\PY{p}{(}\PY{n}{s}\PY{p}{)}
             \PY{k}{else}\PY{p}{:}
                 \PY{n}{angle} \PY{o}{=} \PY{l+m+mi}{0}
                 \PY{n}{width}\PY{p}{,} \PY{n}{height} \PY{o}{=} \PY{l+m+mi}{2} \PY{o}{*} \PY{n}{np}\PY{o}{.}\PY{n}{sqrt}\PY{p}{(}\PY{n}{covariance}\PY{p}{)}
             
             \PY{c+c1}{\PYZsh{} Draw the Ellipse}
             \PY{k}{for} \PY{n}{nsig} \PY{o+ow}{in} \PY{n+nb}{range}\PY{p}{(}\PY{l+m+mi}{1}\PY{p}{,} \PY{l+m+mi}{4}\PY{p}{)}\PY{p}{:}
                 \PY{n}{ax}\PY{o}{.}\PY{n}{add\PYZus{}patch}\PY{p}{(}\PY{n}{Ellipse}\PY{p}{(}\PY{n}{position}\PY{p}{,} \PY{n}{nsig} \PY{o}{*} \PY{n}{width}\PY{p}{,} \PY{n}{nsig} \PY{o}{*} \PY{n}{height}\PY{p}{,}
                                      \PY{n}{angle}\PY{p}{,} \PY{o}{*}\PY{o}{*}\PY{n}{kwargs}\PY{p}{)}\PY{p}{)}
                 
         \PY{k}{def} \PY{n+nf}{plot\PYZus{}gmm}\PY{p}{(}\PY{n}{gmm}\PY{p}{,} \PY{n}{X}\PY{p}{,} \PY{n}{label}\PY{o}{=}\PY{k+kc}{True}\PY{p}{,} \PY{n}{ax}\PY{o}{=}\PY{k+kc}{None}\PY{p}{)}\PY{p}{:}
             \PY{n}{ax} \PY{o}{=} \PY{n}{ax} \PY{o+ow}{or} \PY{n}{plt}\PY{o}{.}\PY{n}{gca}\PY{p}{(}\PY{p}{)}
             \PY{n}{labels} \PY{o}{=} \PY{n}{gmm}\PY{o}{.}\PY{n}{fit}\PY{p}{(}\PY{n}{X}\PY{p}{)}\PY{o}{.}\PY{n}{predict}\PY{p}{(}\PY{n}{X}\PY{p}{)}
             \PY{k}{if} \PY{n}{label}\PY{p}{:}
                 \PY{n}{ax}\PY{o}{.}\PY{n}{scatter}\PY{p}{(}\PY{n}{X}\PY{p}{[}\PY{p}{:}\PY{p}{,} \PY{l+m+mi}{0}\PY{p}{]}\PY{p}{,} \PY{n}{X}\PY{p}{[}\PY{p}{:}\PY{p}{,} \PY{l+m+mi}{1}\PY{p}{]}\PY{p}{,} \PY{n}{c}\PY{o}{=}\PY{n}{labels}\PY{p}{,} \PY{n}{s}\PY{o}{=}\PY{l+m+mi}{40}\PY{p}{,} \PY{n}{cmap}\PY{o}{=}\PY{l+s+s1}{\PYZsq{}}\PY{l+s+s1}{viridis}\PY{l+s+s1}{\PYZsq{}}\PY{p}{,} \PY{n}{zorder}\PY{o}{=}\PY{l+m+mi}{2}\PY{p}{)}
             \PY{k}{else}\PY{p}{:}
                 \PY{n}{ax}\PY{o}{.}\PY{n}{scatter}\PY{p}{(}\PY{n}{X}\PY{p}{[}\PY{p}{:}\PY{p}{,} \PY{l+m+mi}{0}\PY{p}{]}\PY{p}{,} \PY{n}{X}\PY{p}{[}\PY{p}{:}\PY{p}{,} \PY{l+m+mi}{1}\PY{p}{]}\PY{p}{,} \PY{n}{s}\PY{o}{=}\PY{l+m+mi}{40}\PY{p}{,} \PY{n}{zorder}\PY{o}{=}\PY{l+m+mi}{2}\PY{p}{)}
             \PY{n}{ax}\PY{o}{.}\PY{n}{axis}\PY{p}{(}\PY{l+s+s1}{\PYZsq{}}\PY{l+s+s1}{equal}\PY{l+s+s1}{\PYZsq{}}\PY{p}{)}
             
             \PY{n}{w\PYZus{}factor} \PY{o}{=} \PY{l+m+mf}{0.2} \PY{o}{/} \PY{n}{gmm}\PY{o}{.}\PY{n}{weights\PYZus{}}\PY{o}{.}\PY{n}{max}\PY{p}{(}\PY{p}{)}
             \PY{k}{for} \PY{n}{pos}\PY{p}{,} \PY{n}{covar}\PY{p}{,} \PY{n}{w} \PY{o+ow}{in} \PY{n+nb}{zip}\PY{p}{(}\PY{n}{gmm}\PY{o}{.}\PY{n}{means\PYZus{}}\PY{p}{,} \PY{n}{gmm}\PY{o}{.}\PY{n}{covariances\PYZus{}}\PY{p}{,} \PY{n}{gmm}\PY{o}{.}\PY{n}{weights\PYZus{}}\PY{p}{)}\PY{p}{:}
                 \PY{n}{draw\PYZus{}ellipse}\PY{p}{(}\PY{n}{pos}\PY{p}{,} \PY{n}{covar}\PY{p}{,} \PY{n}{alpha}\PY{o}{=}\PY{n}{w} \PY{o}{*} \PY{n}{w\PYZus{}factor}\PY{p}{)}
\end{Verbatim}

    \begin{Verbatim}[commandchars=\\\{\}]
{\color{incolor}In [{\color{incolor}68}]:} \PY{n}{gmm} \PY{o}{=} \PY{n}{GMM}\PY{p}{(}\PY{n}{n\PYZus{}components}\PY{o}{=}\PY{l+m+mi}{4}\PY{p}{,} \PY{n}{random\PYZus{}state}\PY{o}{=}\PY{l+m+mi}{42}\PY{p}{)}
         \PY{n}{plot\PYZus{}gmm}\PY{p}{(}\PY{n}{gmm}\PY{p}{,} \PY{n}{X}\PY{p}{)}
\end{Verbatim}

    \begin{center}
    \adjustimage{max size={0.9\linewidth}{0.9\paperheight}}{Clustering_files/Clustering_62_0.png}
    \end{center}
    { \hspace*{\fill} \\}
    
    Similarly, we can use the GMM approach to fit our stretched dataset;
allowing for a full covariance the model will fit even very oblong,
stretched-out clusters:

    \begin{Verbatim}[commandchars=\\\{\}]
{\color{incolor}In [{\color{incolor}69}]:} \PY{n}{gmm} \PY{o}{=} \PY{n}{GMM}\PY{p}{(}\PY{n}{n\PYZus{}components}\PY{o}{=}\PY{l+m+mi}{4}\PY{p}{,} \PY{n}{covariance\PYZus{}type}\PY{o}{=}\PY{l+s+s1}{\PYZsq{}}\PY{l+s+s1}{full}\PY{l+s+s1}{\PYZsq{}}\PY{p}{,} \PY{n}{random\PYZus{}state}\PY{o}{=}\PY{l+m+mi}{42}\PY{p}{)}
         \PY{n}{plot\PYZus{}gmm}\PY{p}{(}\PY{n}{gmm}\PY{p}{,} \PY{n}{X\PYZus{}stretched}\PY{p}{)}
\end{Verbatim}

    \begin{center}
    \adjustimage{max size={0.9\linewidth}{0.9\paperheight}}{Clustering_files/Clustering_64_0.png}
    \end{center}
    { \hspace*{\fill} \\}
    
    This makes clear that GMM addresses the two main practical issues with
k-means encountered before.

    \hypertarget{choosing-the-covariance-type}{%
\paragraph{Choosing the covariance
type}\label{choosing-the-covariance-type}}

    If we look at the details of the preceding fits, we will see that the
covariance\_type option was set differently within each. This
hyperparameter controls the degrees of freedom in the shape of each
cluster; it is essential to set this carefully for any given problem.
The default is covariance\_type=``diag'', which means that the size of
the cluster along each dimension can be set independently, with the
resulting ellipse constrained to align with the axes. A slightly simpler
and faster model is covariance\_type=``spherical'', which constrains the
shape of the cluster such that all dimensions are equal. The resulting
clustering will have similar characteristics to that of k-means, though
it is not entirely equivalent. A more complicated and computationally
expensive model (especially as the number of dimensions grows) is to use
covariance\_type=``full'', which allows each cluster to be modeled as an
ellipse with arbitrary orientation.

    Though GMM is often categorized as a clustering algorithm, fundamentally
it is an algorithm for density estimation. That is to say, the result of
a GMM fit to some data is technically not a clustering model, but a
generative probabilistic model describing the distribution of the data.

As an example, consider some data generated from Scikit-Learn's
make\_moons function. If we try to fit this with a two-component GMM
viewed as a clustering model, the results are not particularly useful:

    \begin{Verbatim}[commandchars=\\\{\}]
{\color{incolor}In [{\color{incolor}70}]:} \PY{n}{Xmoon}\PY{p}{,} \PY{n}{ymoon} \PY{o}{=} \PY{n}{make\PYZus{}moons}\PY{p}{(}\PY{l+m+mi}{200}\PY{p}{,} \PY{n}{noise}\PY{o}{=}\PY{o}{.}\PY{l+m+mi}{05}\PY{p}{,} \PY{n}{random\PYZus{}state}\PY{o}{=}\PY{l+m+mi}{0}\PY{p}{)}
         \PY{n}{plt}\PY{o}{.}\PY{n}{scatter}\PY{p}{(}\PY{n}{Xmoon}\PY{p}{[}\PY{p}{:}\PY{p}{,} \PY{l+m+mi}{0}\PY{p}{]}\PY{p}{,} \PY{n}{Xmoon}\PY{p}{[}\PY{p}{:}\PY{p}{,} \PY{l+m+mi}{1}\PY{p}{]}\PY{p}{)}\PY{p}{;}
         \PY{n}{gmm2} \PY{o}{=} \PY{n}{GMM}\PY{p}{(}\PY{n}{n\PYZus{}components}\PY{o}{=}\PY{l+m+mi}{2}\PY{p}{,} \PY{n}{covariance\PYZus{}type}\PY{o}{=}\PY{l+s+s1}{\PYZsq{}}\PY{l+s+s1}{full}\PY{l+s+s1}{\PYZsq{}}\PY{p}{,} \PY{n}{random\PYZus{}state}\PY{o}{=}\PY{l+m+mi}{0}\PY{p}{)}
         \PY{n}{plot\PYZus{}gmm}\PY{p}{(}\PY{n}{gmm2}\PY{p}{,} \PY{n}{Xmoon}\PY{p}{)}
\end{Verbatim}

    \begin{center}
    \adjustimage{max size={0.9\linewidth}{0.9\paperheight}}{Clustering_files/Clustering_69_0.png}
    \end{center}
    { \hspace*{\fill} \\}
    
    But if we instead use many more components and ignore the cluster
labels, we find a fit that is much closer to the input data:

    \begin{Verbatim}[commandchars=\\\{\}]
{\color{incolor}In [{\color{incolor}71}]:} \PY{n}{gmm16} \PY{o}{=} \PY{n}{GMM}\PY{p}{(}\PY{n}{n\PYZus{}components}\PY{o}{=}\PY{l+m+mi}{16}\PY{p}{,} \PY{n}{covariance\PYZus{}type}\PY{o}{=}\PY{l+s+s1}{\PYZsq{}}\PY{l+s+s1}{full}\PY{l+s+s1}{\PYZsq{}}\PY{p}{,} \PY{n}{random\PYZus{}state}\PY{o}{=}\PY{l+m+mi}{0}\PY{p}{)}
         \PY{n}{plot\PYZus{}gmm}\PY{p}{(}\PY{n}{gmm16}\PY{p}{,} \PY{n}{Xmoon}\PY{p}{,} \PY{n}{label}\PY{o}{=}\PY{k+kc}{False}\PY{p}{)}
\end{Verbatim}

    \begin{center}
    \adjustimage{max size={0.9\linewidth}{0.9\paperheight}}{Clustering_files/Clustering_71_0.png}
    \end{center}
    { \hspace*{\fill} \\}
    
    Advantage:\\
1.Gives extremely useful result for the real world data set.

Disadvantage:\\
1.Algorithm is highly complex in nature.

    \hypertarget{how-to-choose-initial-points-of-k-clusters-in-k-means-clustering}{%
\subparagraph{How to choose initial points of k clusters in k-means
clustering?}\label{how-to-choose-initial-points-of-k-clusters-in-k-means-clustering}}

    Elbow method is used to determine the optimal value of K to perform the
K-Means Clustering Algorithm. The basic idea behind this method is that
it plots the various values of cost with changing k. As the value of K
increases, there will be fewer elements in the cluster. So average
distortion will decrease. The lesser number of elements means closer to
the centroid. So, the point where this distortion declines the most is
the elbow point. Below is the Python implementation:

    \begin{Verbatim}[commandchars=\\\{\}]
{\color{incolor}In [{\color{incolor}72}]:} \PY{k+kn}{import} \PY{n+nn}{matplotlib}\PY{n+nn}{.}\PY{n+nn}{pyplot} \PY{k}{as} \PY{n+nn}{plt}  
         \PY{k+kn}{from} \PY{n+nn}{matplotlib} \PY{k}{import} \PY{n}{style} 
         \PY{k+kn}{from} \PY{n+nn}{sklearn}\PY{n+nn}{.}\PY{n+nn}{cluster} \PY{k}{import} \PY{n}{KMeans} 
         \PY{k+kn}{from} \PY{n+nn}{sklearn}\PY{n+nn}{.}\PY{n+nn}{datasets}\PY{n+nn}{.}\PY{n+nn}{samples\PYZus{}generator} \PY{k}{import} \PY{n}{make\PYZus{}blobs} 
           
         \PY{n}{style}\PY{o}{.}\PY{n}{use}\PY{p}{(}\PY{l+s+s2}{\PYZdq{}}\PY{l+s+s2}{fivethirtyeight}\PY{l+s+s2}{\PYZdq{}}\PY{p}{)} 
           
         \PY{c+c1}{\PYZsh{} make\PYZus{}blobs() is used to generate sample points }
         \PY{c+c1}{\PYZsh{} around c centers (randomly chosen) }
         \PY{n}{X}\PY{p}{,} \PY{n}{y} \PY{o}{=} \PY{n}{make\PYZus{}blobs}\PY{p}{(}\PY{n}{n\PYZus{}samples} \PY{o}{=} \PY{l+m+mi}{100}\PY{p}{,} \PY{n}{centers} \PY{o}{=} \PY{l+m+mi}{4}\PY{p}{,}  
                         \PY{n}{cluster\PYZus{}std} \PY{o}{=} \PY{l+m+mi}{1}\PY{p}{,} \PY{n}{n\PYZus{}features} \PY{o}{=} \PY{l+m+mi}{2}\PY{p}{)} 
                           
         \PY{n}{plt}\PY{o}{.}\PY{n}{scatter}\PY{p}{(}\PY{n}{X}\PY{p}{[}\PY{p}{:}\PY{p}{,} \PY{l+m+mi}{0}\PY{p}{]}\PY{p}{,} \PY{n}{X}\PY{p}{[}\PY{p}{:}\PY{p}{,} \PY{l+m+mi}{1}\PY{p}{]}\PY{p}{,} \PY{n}{s} \PY{o}{=} \PY{l+m+mi}{30}\PY{p}{,} \PY{n}{color} \PY{o}{=}\PY{l+s+s1}{\PYZsq{}}\PY{l+s+s1}{b}\PY{l+s+s1}{\PYZsq{}}\PY{p}{)} 
           
         \PY{c+c1}{\PYZsh{} label the axes }
         \PY{n}{plt}\PY{o}{.}\PY{n}{xlabel}\PY{p}{(}\PY{l+s+s1}{\PYZsq{}}\PY{l+s+s1}{X}\PY{l+s+s1}{\PYZsq{}}\PY{p}{)} 
         \PY{n}{plt}\PY{o}{.}\PY{n}{ylabel}\PY{p}{(}\PY{l+s+s1}{\PYZsq{}}\PY{l+s+s1}{Y}\PY{l+s+s1}{\PYZsq{}}\PY{p}{)} 
           
         \PY{n}{plt}\PY{o}{.}\PY{n}{show}\PY{p}{(}\PY{p}{)} 
         \PY{n}{plt}\PY{o}{.}\PY{n}{clf}\PY{p}{(}\PY{p}{)} \PY{c+c1}{\PYZsh{} clear the figure }
\end{Verbatim}

    \begin{center}
    \adjustimage{max size={0.9\linewidth}{0.9\paperheight}}{Clustering_files/Clustering_75_0.png}
    \end{center}
    { \hspace*{\fill} \\}
    
    
    \begin{verbatim}
<Figure size 432x288 with 0 Axes>
    \end{verbatim}

    
    \begin{Verbatim}[commandchars=\\\{\}]
{\color{incolor}In [{\color{incolor}73}]:} \PY{n}{cost} \PY{o}{=}\PY{p}{[}\PY{p}{]} 
         \PY{k}{for} \PY{n}{i} \PY{o+ow}{in} \PY{n+nb}{range}\PY{p}{(}\PY{l+m+mi}{1}\PY{p}{,} \PY{l+m+mi}{11}\PY{p}{)}\PY{p}{:} 
             \PY{n}{KM} \PY{o}{=} \PY{n}{KMeans}\PY{p}{(}\PY{n}{n\PYZus{}clusters} \PY{o}{=} \PY{n}{i}\PY{p}{,} \PY{n}{max\PYZus{}iter} \PY{o}{=} \PY{l+m+mi}{500}\PY{p}{)} 
             \PY{n}{KM}\PY{o}{.}\PY{n}{fit}\PY{p}{(}\PY{n}{X}\PY{p}{)} 
               
             \PY{c+c1}{\PYZsh{} calculates squared error }
             \PY{c+c1}{\PYZsh{} for the clustered points }
             \PY{n}{cost}\PY{o}{.}\PY{n}{append}\PY{p}{(}\PY{n}{KM}\PY{o}{.}\PY{n}{inertia\PYZus{}}\PY{p}{)}      
           
         \PY{c+c1}{\PYZsh{} plot the cost against K values }
         \PY{n}{plt}\PY{o}{.}\PY{n}{plot}\PY{p}{(}\PY{n+nb}{range}\PY{p}{(}\PY{l+m+mi}{1}\PY{p}{,} \PY{l+m+mi}{11}\PY{p}{)}\PY{p}{,} \PY{n}{cost}\PY{p}{,} \PY{n}{color} \PY{o}{=}\PY{l+s+s1}{\PYZsq{}}\PY{l+s+s1}{g}\PY{l+s+s1}{\PYZsq{}}\PY{p}{,} \PY{n}{linewidth} \PY{o}{=}\PY{l+s+s1}{\PYZsq{}}\PY{l+s+s1}{3}\PY{l+s+s1}{\PYZsq{}}\PY{p}{)} 
         \PY{n}{plt}\PY{o}{.}\PY{n}{xlabel}\PY{p}{(}\PY{l+s+s2}{\PYZdq{}}\PY{l+s+s2}{Value of K}\PY{l+s+s2}{\PYZdq{}}\PY{p}{)} 
         \PY{n}{plt}\PY{o}{.}\PY{n}{ylabel}\PY{p}{(}\PY{l+s+s2}{\PYZdq{}}\PY{l+s+s2}{Sqaured Error (Cost)}\PY{l+s+s2}{\PYZdq{}}\PY{p}{)} 
         \PY{n}{plt}\PY{o}{.}\PY{n}{show}\PY{p}{(}\PY{p}{)} \PY{c+c1}{\PYZsh{} clear the plot }
           
         \PY{c+c1}{\PYZsh{} the point of the elbow is the  }
         \PY{c+c1}{\PYZsh{} most optimal value for choosing k }
\end{Verbatim}

    \begin{center}
    \adjustimage{max size={0.9\linewidth}{0.9\paperheight}}{Clustering_files/Clustering_76_0.png}
    \end{center}
    { \hspace*{\fill} \\}
    
    In this case the optimal value for k would be 4. (Observable from the
scattered points).


    % Add a bibliography block to the postdoc
    
    
    
    \end{document}
